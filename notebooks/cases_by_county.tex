\PassOptionsToPackage{unicode=true}{hyperref} % options for packages loaded elsewhere
\PassOptionsToPackage{hyphens}{url}
%
\documentclass[]{article}
\usepackage{lmodern}
\usepackage{amssymb,amsmath}
\usepackage{ifxetex,ifluatex}
\usepackage{fixltx2e} % provides \textsubscript
\ifnum 0\ifxetex 1\fi\ifluatex 1\fi=0 % if pdftex
  \usepackage[T1]{fontenc}
  \usepackage[utf8]{inputenc}
  \usepackage{textcomp} % provides euro and other symbols
\else % if luatex or xelatex
  \usepackage{unicode-math}
  \defaultfontfeatures{Ligatures=TeX,Scale=MatchLowercase}
\fi
% use upquote if available, for straight quotes in verbatim environments
\IfFileExists{upquote.sty}{\usepackage{upquote}}{}
% use microtype if available
\IfFileExists{microtype.sty}{%
\usepackage[]{microtype}
\UseMicrotypeSet[protrusion]{basicmath} % disable protrusion for tt fonts
}{}
\IfFileExists{parskip.sty}{%
\usepackage{parskip}
}{% else
\setlength{\parindent}{0pt}
\setlength{\parskip}{6pt plus 2pt minus 1pt}
}
\usepackage{hyperref}
\hypersetup{
            pdftitle={COVID-19 Confirmed Cases by Colorado County},
            pdfauthor={Joseph Tuccillo},
            pdfborder={0 0 0},
            breaklinks=true}
\urlstyle{same}  % don't use monospace font for urls
\usepackage[margin=1in]{geometry}
\usepackage{color}
\usepackage{fancyvrb}
\newcommand{\VerbBar}{|}
\newcommand{\VERB}{\Verb[commandchars=\\\{\}]}
\DefineVerbatimEnvironment{Highlighting}{Verbatim}{commandchars=\\\{\}}
% Add ',fontsize=\small' for more characters per line
\usepackage{framed}
\definecolor{shadecolor}{RGB}{248,248,248}
\newenvironment{Shaded}{\begin{snugshade}}{\end{snugshade}}
\newcommand{\AlertTok}[1]{\textcolor[rgb]{0.94,0.16,0.16}{#1}}
\newcommand{\AnnotationTok}[1]{\textcolor[rgb]{0.56,0.35,0.01}{\textbf{\textit{#1}}}}
\newcommand{\AttributeTok}[1]{\textcolor[rgb]{0.77,0.63,0.00}{#1}}
\newcommand{\BaseNTok}[1]{\textcolor[rgb]{0.00,0.00,0.81}{#1}}
\newcommand{\BuiltInTok}[1]{#1}
\newcommand{\CharTok}[1]{\textcolor[rgb]{0.31,0.60,0.02}{#1}}
\newcommand{\CommentTok}[1]{\textcolor[rgb]{0.56,0.35,0.01}{\textit{#1}}}
\newcommand{\CommentVarTok}[1]{\textcolor[rgb]{0.56,0.35,0.01}{\textbf{\textit{#1}}}}
\newcommand{\ConstantTok}[1]{\textcolor[rgb]{0.00,0.00,0.00}{#1}}
\newcommand{\ControlFlowTok}[1]{\textcolor[rgb]{0.13,0.29,0.53}{\textbf{#1}}}
\newcommand{\DataTypeTok}[1]{\textcolor[rgb]{0.13,0.29,0.53}{#1}}
\newcommand{\DecValTok}[1]{\textcolor[rgb]{0.00,0.00,0.81}{#1}}
\newcommand{\DocumentationTok}[1]{\textcolor[rgb]{0.56,0.35,0.01}{\textbf{\textit{#1}}}}
\newcommand{\ErrorTok}[1]{\textcolor[rgb]{0.64,0.00,0.00}{\textbf{#1}}}
\newcommand{\ExtensionTok}[1]{#1}
\newcommand{\FloatTok}[1]{\textcolor[rgb]{0.00,0.00,0.81}{#1}}
\newcommand{\FunctionTok}[1]{\textcolor[rgb]{0.00,0.00,0.00}{#1}}
\newcommand{\ImportTok}[1]{#1}
\newcommand{\InformationTok}[1]{\textcolor[rgb]{0.56,0.35,0.01}{\textbf{\textit{#1}}}}
\newcommand{\KeywordTok}[1]{\textcolor[rgb]{0.13,0.29,0.53}{\textbf{#1}}}
\newcommand{\NormalTok}[1]{#1}
\newcommand{\OperatorTok}[1]{\textcolor[rgb]{0.81,0.36,0.00}{\textbf{#1}}}
\newcommand{\OtherTok}[1]{\textcolor[rgb]{0.56,0.35,0.01}{#1}}
\newcommand{\PreprocessorTok}[1]{\textcolor[rgb]{0.56,0.35,0.01}{\textit{#1}}}
\newcommand{\RegionMarkerTok}[1]{#1}
\newcommand{\SpecialCharTok}[1]{\textcolor[rgb]{0.00,0.00,0.00}{#1}}
\newcommand{\SpecialStringTok}[1]{\textcolor[rgb]{0.31,0.60,0.02}{#1}}
\newcommand{\StringTok}[1]{\textcolor[rgb]{0.31,0.60,0.02}{#1}}
\newcommand{\VariableTok}[1]{\textcolor[rgb]{0.00,0.00,0.00}{#1}}
\newcommand{\VerbatimStringTok}[1]{\textcolor[rgb]{0.31,0.60,0.02}{#1}}
\newcommand{\WarningTok}[1]{\textcolor[rgb]{0.56,0.35,0.01}{\textbf{\textit{#1}}}}
\usepackage{graphicx,grffile}
\makeatletter
\def\maxwidth{\ifdim\Gin@nat@width>\linewidth\linewidth\else\Gin@nat@width\fi}
\def\maxheight{\ifdim\Gin@nat@height>\textheight\textheight\else\Gin@nat@height\fi}
\makeatother
% Scale images if necessary, so that they will not overflow the page
% margins by default, and it is still possible to overwrite the defaults
% using explicit options in \includegraphics[width, height, ...]{}
\setkeys{Gin}{width=\maxwidth,height=\maxheight,keepaspectratio}
\setlength{\emergencystretch}{3em}  % prevent overfull lines
\providecommand{\tightlist}{%
  \setlength{\itemsep}{0pt}\setlength{\parskip}{0pt}}
\setcounter{secnumdepth}{0}
% Redefines (sub)paragraphs to behave more like sections
\ifx\paragraph\undefined\else
\let\oldparagraph\paragraph
\renewcommand{\paragraph}[1]{\oldparagraph{#1}\mbox{}}
\fi
\ifx\subparagraph\undefined\else
\let\oldsubparagraph\subparagraph
\renewcommand{\subparagraph}[1]{\oldsubparagraph{#1}\mbox{}}
\fi

% set default figure placement to htbp
\makeatletter
\def\fps@figure{htbp}
\makeatother

\usepackage{etoolbox}
\makeatletter
\providecommand{\subtitle}[1]{% add subtitle to \maketitle
  \apptocmd{\@title}{\par {\large #1 \par}}{}{}
}
\makeatother
% https://github.com/rstudio/rmarkdown/issues/337
\let\rmarkdownfootnote\footnote%
\def\footnote{\protect\rmarkdownfootnote}

% https://github.com/rstudio/rmarkdown/pull/252
\usepackage{titling}
\setlength{\droptitle}{-2em}

\pretitle{\vspace{\droptitle}\centering\huge}
\posttitle{\par}

\preauthor{\centering\large\emph}
\postauthor{\par}

\predate{\centering\large\emph}
\postdate{\par}

\title{COVID-19 Confirmed Cases by Colorado County}
\author{Joseph Tuccillo}
\date{09 April, 2020}

\begin{document}
\maketitle

\begin{Shaded}
\begin{Highlighting}[]
\NormalTok{knitr}\OperatorTok{::}\NormalTok{opts_chunk}\OperatorTok{$}\KeywordTok{set}\NormalTok{(}\DataTypeTok{fig.path=}\StringTok{'figs/'}\NormalTok{,}
                      \DataTypeTok{fig.pos=}\StringTok{'h'}\NormalTok{,}
                      \CommentTok{# out.extra='',}
                      \DataTypeTok{fig.ext =} \StringTok{'pdf'}\NormalTok{,}
                      \CommentTok{# fig.ext = 'png',}
                      \CommentTok{# dpi = 320,}
                      \DataTypeTok{echo=}\OtherTok{FALSE}\NormalTok{,}
                      \DataTypeTok{warning=}\OtherTok{FALSE}\NormalTok{,}
                      \DataTypeTok{message=}\OtherTok{FALSE}\NormalTok{)}
\NormalTok{knitr}\OperatorTok{::}\NormalTok{opts_knit}\OperatorTok{$}\KeywordTok{set}\NormalTok{(}\DataTypeTok{root.dir =} \StringTok{'../'}\NormalTok{)}
\end{Highlighting}
\end{Shaded}

\hypertarget{daily-cases-by-colorado-county}{%
\subsection{Daily Cases by Colorado
County}\label{daily-cases-by-colorado-county}}

\begin{verbatim}
## Using an auto-discovered, cached token.
## To suppress this message, modify your code or options to clearly consent to the use of a cached token.
## See gargle's "Non-interactive auth" vignette for more details:
## https://gargle.r-lib.org/articles/non-interactive-auth.html
## The googledrive package is using a cached token for jotu9073@colorado.edu.
\end{verbatim}

\hypertarget{raw-cases-per-100000-people-by-county}{%
\subsubsection{Raw Cases per 100,000 People by
County}\label{raw-cases-per-100000-people-by-county}}

\includegraphics{figs/daily-cases-100k-1.pdf}

\hypertarget{log-transformed-cases-per-100000-people-by-county}{%
\subsubsection{Log-Transformed Cases per 100,000 People by
County}\label{log-transformed-cases-per-100000-people-by-county}}

\includegraphics{figs/daily-cases-100k-log-1.pdf}

\newpage

\hypertarget{county-confirmed-case-trajectories-experimental}{%
\subsection{County Confirmed Case Trajectories
(Experimental)}\label{county-confirmed-case-trajectories-experimental}}

\hypertarget{methodology}{%
\subsubsection{Methodology}\label{methodology}}

Use Affinity Propagation clustering to group daily reports of confirmed
cases by county based on two criteria:

\begin{enumerate}
\def\labelenumi{\arabic{enumi}.}
\tightlist
\item
  The current rate of confirmed cases per 100,000 people.
\item
  The change in confirmed cases per 100,000 people since the previous
  day.
\end{enumerate}

\begin{itemize}
\item
  March 20, 2020 is used as the intial date, since it is the first day
  at which the change in cases/100k people by county can be measured
  (3/19/2020 marks the first day in which all counties were reporting).
\item
  From the ensemble of daily clusterings, measure the percentage of days
  to date that any two counties shared a cluster label.
\item
  Perform a final clustering (also using Affinity Propagation) to group
  the change trajectories from 3/20/2020 to present.
\end{itemize}

\hypertarget{smoothed-raw-cases-per-100000-people-by-cluster-with-variable-y-axes}{%
\subsection{Smoothed Raw Cases per 100,000 people by cluster with
variable
y-axes}\label{smoothed-raw-cases-per-100000-people-by-cluster-with-variable-y-axes}}

\includegraphics{figs/county-clusters-smooth-raw-1.pdf}

\hypertarget{smoothed-log-transformed-cases-per-100000-people-by-cluster-with-fixed-y-axes}{%
\subsection{Smoothed Log-transformed Cases per 100,000 people by cluster
with fixed
y-axes}\label{smoothed-log-transformed-cases-per-100000-people-by-cluster-with-fixed-y-axes}}

\includegraphics{figs/county-clusters-smooth-log-1.pdf}

\end{document}
